@article{An2002b,
   abstract = {The presence of magnetic background field inhomogeneity (ΔB) may confound quantitative measures of cerebral venous blood volume (vCBV) and cerebral oxygen extraction fraction (MR_OEF) with T2*-based methods. The goal of this study was to correct its effect and obtain more accurate estimates of vCBV and MR_OEF. A 3D high-resolution gradient echo sequence was employed to obtain ΔB maps by two algorithms. The ΔB maps were then used to recover the signal loss in images acquired by a 2D multiecho gradient echo / spin echo sequence. Finally, both quantitative estimates of MR_OEF and vCBV were obtained from the ΔB-corrected 2D multiecho gradient echo / spin echo images. A total of 12 normal subjects were studied. An overestimated vCBV was observed in the brain (4.29 ± 0.78\%) prior to ΔB correction, while the measured vCBV was substantially reduced after ΔB correction. Whole brain vCBV of 2.97 ± 0.44\% and 2.68 ± 0.47\% were obtained by the two different ΔB correction methods, in excellent agreement with the reported results in the literature. Furthermore, when MR_OEF was compared with and without ΔB correction, no significant differences (P = 0.467) were observed. The ability to simultaneously obtain vCBV and MR_OEF noninvasively may have profound clinical implications for the studies of cerebrovascular disease. © 2002 Wiley-Liss, Inc.},
   author = {Hongyu An and Weili Lin},
   doi = {10.1002/MRM.10148},
   issn = {1522-2594},
   issue = {5},
   journal = {Magnetic Resonance in Medicine},
   keywords = {BOLD,background field inhomogeneity,cerebral venous blood volume},
   month = {5},
   pages = {958-966},
   pmid = {11979575},
   publisher = {John Wiley & Sons, Ltd},
   title = {Cerebral oxygen extraction fraction and cerebral venous blood volume measurements using MRI: Effects of magnetic field variation},
   volume = {47},
   url = {https://onlinelibrary.wiley.com/doi/full/10.1002/mrm.10148 https://onlinelibrary.wiley.com/doi/abs/10.1002/mrm.10148 https://onlinelibrary.wiley.com/doi/10.1002/mrm.10148},
   year = {2002},
}